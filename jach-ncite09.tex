\documentclass{acm_proc_article-sp}
\usepackage{url}

\begin{document}


\title{ICS-OS: A Kernel Programming Approach to Teaching Operating System
Concepts\titlenote{ICS-OS is an open source project hosted at
http://code.google.com/p/ics-os/}}

\numberofauthors{1}
\author{
\alignauthor
Joseph Anthony C. Hermocilla\\
       \affaddr{Institute of Computer Science}\\
       \affaddr{College of Arts and Sciences}\\
       \affaddr{University of the Philippines Los Ba\~nos}\\
       \affaddr{College 4031, Laguna, Philippines}\\
       \email{jachermocilla@uplb.edu.ph}
}
\date{02 September 2009}

\maketitle
\begin{abstract}
Traditional approaches to teaching operating systems require students to
develop simulations and user space applications. An alternative
is to let them modify parts of an actual operating system and see their
programs run at kernel space. However, this is difficult to 
achieve using modern real-world operating systems partly because of the complex
and large source code base. This paper presents ICS-OS and the experiences and
results of using it for teaching an undergraduate operating systems course. 
ICS-OS is based on the DEX-OS kernel which has a smaller source code base
compared to mainstream operating systems, making it ideal for instruction. 
The students were able to demonstrate a deeper understanding
of how a real operating system works by their succesful implementation of
projects to enhance and extend ICS-OS.
\end{abstract}

\category{D.4.7}{Operating Systems}{Organization and Design} \\
\category{K.3.2}{Computer and Information Science Education}
{Computer Science Education}

\keywords{Operating systems, computer science education, kernel} 

\section{Introduction}
Operating systems is a core knowledge area in computer science education
emphasized by the ACM and IEEE review task force for the computer science 
curriculum. Traditional approaches to teaching operating systems to 
undergraduates, as in the case at the author's institution, do not involve 
programming the components of an actual operating system that can run on 
real hardware. Instead, simulations are used or user space application 
development are done. Students, however, are more interested in writing code
that runs at the kernel space, internal to the operating system itself. They
want to concretize the abstract concepts in operating systems through kernel
code. To achieve this, either the students can build an operating system from
scratch\cite{black:osfs} or modify an existing one.

Building an operating system from scratch is not the best option since a course
is usually offered for a semester and there is not enough time to finish. In 
addition, the prerequisite knowledge needed to make an operating system may have
not been acquired by the students yet. To write an operating system from 
scratch, one has to have knowledge of the processor architecture, assembly
language, data structures, algorithms, and low-level C programming.

Modifying an actual operating system that runs on real hardware is a more
viable alternative. However, the choice of the operating system to use is
still an issue. In the past, several instructional operating systems have been
proposed and developed. The next section briefly reviews some of them 
to highlight their strengths and weaknesses.

Two possible criteria for choosing the operating system to use are completeness 
and size of the source code base. An instructional operating system that does
not implement high level abstractions like process management, memory 
management, and filesystems will unlikely be a good choice because
of the missing features. On the other hand, an operating system with several 
thousands of lines of code and a complicated source directory structure will
confuse students and will take more time to understand. Thus there should
be a right balance between completeness and code size.

Recent developments in hardware emulation and virtualization has also made it 
easier to work with real-world operating systems. Testing a kernel need not
require a reboot of the development machine for testing. Unnecessary 
boostrapping is no longer needed since the test machine is a software
application running on the development machine itself.
 
The delivery of an operating systems course is usually through a
lecture and a laboratory component. A popular textbook used by instructors in
the lecture is the dinosaur book by Silberschatz and Galvin 
\cite{silberschatz:osc}. Typical laboratory activities involves learning to 
use a unix-based operating system, developing simulations for different process
scheduling algorithms, understanding the fork() and exec() system calls, 
programming using user level threads, and implementing interprocess
communications. All of these however are in user space and do not involve 
writing kernel code.

This paper presents ICS-OS and the experiences and results of using it for 
teaching an undergraduate operating systems course, specifically in the
laboratory, at the Institute of Computer Science, University of the Philippines
Los Ba\~nos. The students who took the course are in their third year and
has completed the data structures and assembly language prerequisites.

%%\cite{silberschatz:osc},\cite{tanenbaum:osdai},\cite{dayo:dexos},
%%\cite{claypool:fossil},\cite{christopher:nachos},\cite{gary:nachos},
%%\cite{tanenbaum:minix},\cite{black:osfs},\cite{hovemeyer:geekos},
%%\cite{anderson:survey}


\section{Related Work}
Several instructional operating systems have been proposed and developed in 
the past\cite{anderson:survey}. This section briefly describes the popular ones to highlight
their strengths and weaknesses. Mainstream operating systems such as linux and
BSD were not considered because they are too complicated already.

\subsection{Minix}
Minix\cite{tanenbaum:minix} has been around for several years already and has 
been the running example in a textbook\cite{tanenbaum:osdai}. It is based
on Unix, POSIX compliant, and runs on real hardware. Since its initial release,
Minix has grown in size in terms of source code as well as in complexity 
which makes it difficult to use for teaching given advanced features like 
networking support.

\subsection{Nachos}
Nachos\cite{christopher:nachos} is instructional software for teaching 
undergraduate, and potentially graduate, level operating systems courses.
The Nachos kernel and hardware simulator runs as Unix processes implemented
in C++. A success story in using Nachos for teaching can be found in 
\cite{gary:nachos}. The fact that Nachos runs as a Unix process limits 
its appeal because it cannot run on a real hardware and is dependent on the
host.

\subsection{GeekOS}
GeekOS\cite{hovemeyer:geekos} is an instructional operating system that
runs on real hardware. The design goals for GeekOS are simplicity, realism,
and understandability. Its main features include interrupt handling, 
heap memory allocator, time-sliced kernel threads with static priority
scheduling, mutexes and condition variables, user mode with segmentation-based
memory protection, and device drivers for keyboard and VGA.


\subsection{DEX-OS}
DEX-OS\cite{dayo:dexos,dexos:site} is an educational operating system based on
an aspect-oriented approach to address cross-cutting concerns in operating 
systems design and implementation.The main components of DEX-OS are memory 
manager, process manager, device manager, and virtual file system. DEX-OS runs
on the Intel 386 platform in 32-bit protected mode.

\section{ICS-OS Overview}
Similar to the linux philosophy, one can think of ICS-OS as a 
\textit{distribution} that is based on the DEX-OS kernel. ICS-OS provides 
an easy environment for learning operating systems by kernel programming. 
A set of tools and utilities were implemented to make it easy for students
to write and modify kernel code. 

ICS-OS is generally divided into two main components, the kernel and user 
applications. The kernel is loaded at bootup using GRUB as the bootloader.
Figure 1 shows ICS-OS booting. Essentially, the kernel stays in the main memory
of the computer until shutdown. Since GRUB supports compressed kernel images, 
the size of the kernel binary is further reduced, correspondingly the 
distribution size. ICS-OS can thus fit in a single floppy disk. User 
applications, on the other hand, are loaded by the user manually through the
shell or by initialization scripts. The shell in ICS-OS is implemented as part 
of the kernel (Figure 2).

These two components provide a diverse area for experimentation by the students.
They can work on the kernel or develop user applications. A software 
development kit and a minimal C standard library is provided for programming 
applications. Figure 3 shows the \textit{ls} command being executed by the 
shell.


\begin{figure}
\centering
\epsfig{file=booting.eps, height=2in, width=3in}
\caption{Booting ICS-OS.}
\end{figure}

\begin{figure}
\centering
\epsfig{file=shell.eps, height=2in, width=3in}
\caption{ICS-OS shell.}
\end{figure}

\begin{figure}
\centering
\epsfig{file=ls.eps, height=2in, width=3in}
\caption{Executing the \textit{ls} command.}
\end{figure}


\section{Teaching with ICS-OS}

\subsection{Development Environment}
A convenient environment for operating system kernel development is essential.
The compilers, build tools, disk utilities, and emulators should be readily 
available on the development machine. A linux environment can provide all these
tools. Thus, the laboratory computers were reformatted and Ubuntu was installed.
The following additional packages were also added to complete the environment.
\begin{itemize}
 \item Make
 \item GCC/TCC
 \item NASM
 \item Bochs
 \item VirtualBox
 \item Subversion (optional)
\end{itemize}

\subsection{Student Activities}
After the development environment is ready, students were given activities
to familiarize themselves with the source code of ICS-OS.

\subsubsection{Bootloader and Intel 386 Protected Mode}
This activity, although not directly related to ICS-OS, was conducted in order
for the students to understand how a PC boots. They were asked to develop a
simple bootloader using assembly language. The exercise was then extended so
that the CPU goes into the protected mode instead of the default real mode.
This is because modern operating systems run in protected as with the case
of ICS-OS.

\subsubsection{Download, Build, and Test}
The first activity is to download the source code of ICS-OS\cite{icsos:site}.
The source code is distributed in a tar.gz file and via Subversion. 
After extracting, students can  build and test ICS-OS using the commands below.
\begin{verbatim}
(1) $make
(2) $sudo make install
(3) $bochs -q
\end{verbatim}

The first command creates the kernel binary in ELF format.The second command 
builds the distribution floppy image using the loopback device for mounting,
and thus requires adminstrator privileges so the \textit{sudo} command is used.
The third command starts Bochs\cite{bochs:site} which loads ICS-OS (see 
Figure 1).

\subsubsection{Adding a new shell command}
The best way to understand how ICS-OS works is to start with the shell since 
it contains the commands the users can use to access the services provided 
by the operating system. Thus in the next activity,  students modify the kernel,
particularly the shell, by adding an internal command. The students were
directed to navigate to the console source code (kernel/console/console.c). 
They were asked to modify the function console\_execute() by adding the 
following simple code fragment.

\begin{verbatim}
/*--START--*/
if (strcmp(u,"hello")==0)
{
  printf("Hello World command!\n");
}
else
/*--END--*/
\end{verbatim}

The result of this activity is shown in Figure 4.

\begin{figure}
\centering
\epsfig{file=hello.eps, height=2in, width=3in}
\caption{Executing the newly created \textit{hello} internal command.}
\end{figure}


\subsubsection{Developing User Applications}
Application development in ICS-OS is done via a software development kit.
The externally callable functions and system calls are implemented in the file 
sdk/tccsdk.c. User programs are linked against this file to generate the
executable code that is compatible and runnable within ICS-OS. To ease the 
development process, a Makefile was created to perform the build
automatically. An example Makefile is shown below.

\begin{verbatim}
CC=gcc
ICSOS_ROOT=../..
SDK=../../sdk
CFLAGS=-nostdlib -fno-builtin -static
LIBS=$(SDK)/tccsdk.c $(SDK)/libtcc1.c $(SDK)/crt1.c
EXE=hello.exe
$(EXE): hello.c
        $(CC) $(CFLAGS) -o${EXE} hello.c $(LIBS)
install: $(EXE)
        cp $(EXE) $(ICSOS_ROOT)/apps
uninstall:
        rm $(ICSOS_ROOT)/apps/$(EXE)
clean:
        rm $(EXE)
\end{verbatim}

In this actvity, students were asked to develop an external program that can 
be executed from within ICS-OS. They created their program folder within
the contrib/ directory and copied the sample Makefile. The \textit{install}
target in the Makefile will simply copy the executable to the apps/ folder.
It is only after the distribution floppy is created will the application be
available inside ICS-OS. Figure 5 shows the execution of the sample 
application.

\begin{figure}
\centering
\epsfig{file=helloapp.eps, height=2in, width=3in}
\caption{Executing the hello.exe application inside ICS-OS.}
\end{figure}
 

\subsubsection{Student Projects}

\section{Results and Discussion}

\section{Conclusion and Future Work}

\section{Acknowledgments}


\bibliographystyle{abbrv}
\bibliography{jach-ncite09.bib}
\balancecolumns


\end{document}
