\documentclass{acm_proc_article-sp}
\begin{document}

\title{Enhancing Undergraduate Operating Systems Course Through 
A Real Operating System}

\numberofauthors{2}
\author{
\alignauthor
Joseph Anthony C. Hermocilla\\
       \affaddr{Institute of Computer Science}\\
       \affaddr{University of the Philippines Los Ba\~nos}\\
       \affaddr{College 4031, Laguna, Philippines}\\
       \email{jachermocilla@uplb.edu.ph}
\alignauthor
Joseph Emmanuel DL. Dayo\\
       \affaddr{Friendster, Incorporated}\\      
       \email{joseph.dayo@gmail.com}
}
\date{02 September 2009}

\maketitle
\begin{abstract}
Operating systems is a fundamental knowledge area in Computer Science Education
emphasized by the ACM and IEEE review task force for the Computer Science 
Curriculum. An operating system is a piece of software that provides a layer 
of abstraction to make it easy and convenient for users to control the hardware.
The many services and functions supported by an operating system makes it a
rather complex piece of software comprising of several hundreds to thousand
lines of source code. This paper presents the experiences and results of using
ICS-OS for teaching operating systems concepts to undergraduate students at the
Institute of Computer Science, University of the Philippines Los Banos. 
Traditional approaches to teaching operating systems concepts has relied on 
simulations and user space application development because of the large source
code base of currently available open source and commercial operating systems.
However, these approaches are not enough since students do not see their 
programs working at the kernel level. ICS-OS is a fully functional operating 
system, based on DEX-OS, with a small source code base making it ideal for 
teaching. At the end of the semester the course was offered, the students were
able to create interesting projects related to operating systems concepts that
demonstrate their understanding and appreciation.
\end{abstract}

\keywords{operating systems, computer science education} 

\section{Introduction}

\section{Significance}

\section{Related Work}

\subsection{Minix}
Minix has been around for several years already and has been the running 
example in the textbook, Operating Systems: Design and Implementation by 
Andrew Tanenbaum. Since its initial release, Minix has grown in size in terms
of source code as well as in complexity. 

\subsection{Nachos}

\subsection{GeekOS}

GeekOS is another instructional operating system that runs on real hardware. 
It was developed by David Hovemeyer and used for instruction at the Department 
of Computer Science, University of Maryland. The design goals for GeekOS are
simplicity, realism, and understandability. The main features of GeekOS include interrupt handling, heap memory allocator, time-sliced kernel threads with 
static priority scheduling, mutexes and condition variables, user mode 
with segmentation-based memory protection, and device drivers for keyboard 
and VGA.


\subsection{Dex-OS}
Dex-OS was developed by Joseph Emmanuel Dayo. Dex-OS was opensourced under 
the GPL with the original project hosted at SourceForge. Its design is based on
an Aspect Oriented Approach to address cross-cutting concerns in operating 
systems design and implementation. A layered approach was taken based on 
modern operating systems like linux and windows. The following are the main
components of DEX-OS.

\textit{Memory Manager}

The memory manager is responsible for allocating memory to various components
of the operating system, including processes. It is further divided into 
low-level memory manager for setting up the GDT and high-level memory manager
for high-level functions such as malloc() and free().

\textit{Device Manager}

The device manager handles operations related to common devices and peripherals.
Device drivers for keyboard, display, and disk drives are managed by this 
module. Each device being managed is identified by a device id. 

\textit{Process Manager}

The process manager is responsible for handling process management functions
including process creation, process dispatching, process scheduling, process
synchronization, and process termination. 

\textit{Virtual Filesystem}

The Virtual File System (VFS) provides an device-independent abstraction for
various directory and file operations. Functions that are part of the VFS are
implemented by the low-level file system drivers. In particular, DEX-OS 
supports the Microsoft FAT12, FAT16, and FAT32 file systems.

\section{Methodology}

\subsection{Setup a linux development environment}
The original development environment for DEX-OS was in Windows and uses the 
Dev-C++ IDE. Since the laboratory rooms of the students were classes are held
use linux, the first step was to migrate the development development into 
linux. The following steps were taken.

\section{Results and Discussion}

\section{Conclusion}

\bibliographystyle{abbrv}
\bibliography{jach-ncite09.bbl}
\balancecolumns


\end{document}
