\documentclass{acm_proc_article-sp}
\begin{document}

\title{ICS-OS: A Kernel Programming Approach to Teaching Operating System
Concepts}

\numberofauthors{1}
\author{
\alignauthor
Joseph Anthony C. Hermocilla\\
       \affaddr{Institute of Computer Science}\\
       \affaddr{University of the Philippines Los Ba\~nos}\\
       \affaddr{College 4031, Laguna, Philippines}\\
       \email{jachermocilla@uplb.edu.ph}
}
\date{02 September 2009}

\maketitle
\begin{abstract}
Traditional approaches to teaching operating systems require students to
develop simulations and user space applications. An alternative method
is to let them modify parts of an actual operating system and see their
program run at kernel space. However, this is difficult to 
achieve using modern real-world operating systems partly because of the complex
and large source code base. This paper presents ICS-OS and the experiences and
results of using it for teaching an undergraduate operating systems course. 
ICS-OS is based on the DEX-OS kernel which has a smaller source code base
compared to other operating systems making it ideal for instruction. 
The students were able to demonstrate a deeper understanding
of how a real operating system works by their succesful implementation of
projects to enhance and extend ICS-OS.
\end{abstract}

\keywords{Operating systems, computer science education} 

\section{Introduction}
Operating systems is a fundamental knowledge area in computer science education
as emphasized by the ACM and IEEE review task force for the computer science 
curriculum. 

An operating system is a piece of software that provides a layer 
of abstraction to make it easy and convenient for users to control the hardware.

\section{Significance}
Current approaches to teaching operating systems to undergraduates does not
involve programming the components of an actual operating system that can run
on real hardware. Simulations are used or user level application development.
Students, however, are more interested in writing code that can run at the
kernel level instead of just user level. 

\section{Related Work}

\subsection{Minix}
Minix has been around for several years already and has been the running 
example in the textbook, Operating Systems: Design and Implementation by 
Andrew Tanenbaum. Since its initial release, Minix has grown in size in terms
of source code as well as in complexity. 

\subsection{Nachos}
Nachos is instructional software for teaching undergraduate, and potentially
graduate, level operating systems courses. The Nachos distribution
comes with: 

   an overview paper
   simple baseline code for a working operating system
   a simulator for a generic personal computer/workstation
   sample assignments
   a C++ primer (Nachos is written in an easy-to-learn subset of C++, 
     and the primer helps teach C programmers our subset)

The assignments illustrate and explore all areas of modern operating
systems, including threads and concurrency, multiprogramming, 
system calls, virtual memory, software-loaded TLB's, file systems, 
network protocols, remote procedure call, and distributed systems.


\subsection{GeekOS}

GeekOS is another instructional operating system that runs on real hardware. 
It was developed by David Hovemeyer and used for instruction at the Department 
of Computer Science, University of Maryland. The design goals for GeekOS are
simplicity, realism, and understandability. The main features of GeekOS include interrupt handling, heap memory allocator, time-sliced kernel threads with 
static priority scheduling, mutexes and condition variables, user mode 
with segmentation-based memory protection, and device drivers for keyboard 
and VGA.


\subsection{Dex-OS}
Dex-OS was developed by Joseph Emmanuel Dayo. Dex-OS was opensourced under 
the GPL with the original project hosted at SourceForge. Its design is based on
an Aspect Oriented Approach to address cross-cutting concerns in operating 
systems design and implementation. A layered approach was taken based on 
modern operating systems like linux and windows. The following are the main
components of DEX-OS.

\textit{Memory Manager}

The memory manager is responsible for allocating memory to various components
of the operating system, including processes. It is further divided into 
low-level memory manager for setting up the GDT and high-level memory manager
for high-level functions such as malloc() and free().

\textit{Device Manager}

The device manager handles operations related to common devices and peripherals.
Device drivers for keyboard, display, and disk drives are managed by this 
module. Each device being managed is identified by a device id. 

\textit{Process Manager}

The process manager is responsible for handling process management functions
including process creation, process dispatching, process scheduling, process
synchronization, and process termination. 

\textit{Virtual Filesystem}

The Virtual File System (VFS) provides an device-independent abstraction for
various directory and file operations. Functions that are part of the VFS are
implemented by the low-level file system drivers. In particular, DEX-OS 
supports the Microsoft FAT12, FAT16, and FAT32 file systems.

\section{Methodology}

\subsection{Setup a linux development environment}
The original development environment for DEX-OS was in Windows and uses the 
Dev-C++ IDE. Since the laboratory rooms of the students were classes are held
use linux, the first step was to migrate the development environment to 
linux. The following steps were taken.

The Makefile is the primary mechanism to build the ICS-OS kernel and 
generate a distribution floppy. The original Makefile and Linker Script
was converted to generate an ELF kernel image (linux) instead of COFF.


\section{Results and Discussion}

\section{Conclusion}

\bibliographystyle{abbrv}
\bibliography{jach-ncite09.bbl}
\balancecolumns


\end{document}
