% cv-us.tex
% $Id$
%
% LaTeX Curriculum Vitae Template
%
% Copyright (C) 2004-2006 Jason Blevins
%
% You may use use this document as a template to create your own CV
% and you may redistribute the source code freely. No attribution is
% required in any resulting documents, however, I do ask that you
% please leave this notice and the above URL in the source code if you
% choose to redistribute this file.
%
% Jason R. Blevins <jrblevin@sdf.lonestar.org>
% http://jrblevin.freeshell.org
% Durham, December 12, 2006
%
%%---------------------------------------------------------------------------%
%
% Notes:
%
% * Don't forget to change `pdfauthor' and `keywords' in the \hypersetup
%   section below.
%
% * To create a new page use: \newpage \opening
%
% * res.cls includes an \address{} command which can be used up to twice,
%   but my address is too long for the format it uses.
%
% * Alternate documentclass statement to put headings in margin:
%   \documentclass[margin,line,11pt,final]{res}
%
% * Can divide publication/presentation list into subsections by year:
%   \section{\bf\small\hspace{8mm}2006}
%
%%----------------------------------------------------------------------------%

\documentclass[overlapped,line,letterpaper]{res}

\usepackage{ifpdf}

\ifpdf
  \usepackage[pdftex]{hyperref}
\else
  \usepackage[hypertex]{hyperref}
\fi

\hypersetup{
  letterpaper,
  colorlinks,
  urlcolor=black,
  pdfpagemode=none,
  pdftitle={Curriculum Vitae},
  pdfauthor={Joseph Anthony C. Hermocilla},
  pdfcreator={},
  pdfsubject={Curriculum Vitae},
  pdfkeywords={computer science multiagent systems}
}

%%===========================================================================%%

\begin{document}

%---------------------------------------------------------------------------
% Document Specific Customizations

% Make lists without bullets and with no indentation
\setlength{\leftmargini}{0em}
\renewcommand{\labelitemi}{}

% Use large bold font for printed name at top of pages
\renewcommand{\namefont}{\large\textbf}

%---------------------------------------------------------------------------

\name{Joseph Anthony C. Hermocilla}

\begin{resume}

\begin{ncolumn}{2}
  Institute of Computer Science 					& Phone: (049) 536-2313 \\
  College of Arts and Sciences  					& Fax: (049) 536-2302 \\
  University of the Philippines Los Ba\~{n}os   	& {\tt jachermocilla@gmail.com} \\
  College, Los Ba\~{n}os, Laguna 4031				& {\tt \verb+http://jachermocilla.org+} \\
  Philippines		     				    	& \\
\end{ncolumn}

%---------------------------------------------------------------------------

\section{\bf Education}
M.S. Computer Science, University of the Philippines Los Ba\~{n}os, 2008 			\\
B.S. Computer Science, University of the Philippines Los Ba\~{n}os, 2002				\\
High School, Ateneo de Naga University, 1996							\\
Elementary, Naga Parochial School, 1992

%---------------------------------------------------------------------------

\section{\bf Research}

\begin{format}
\title{l}\dates{r}\\
\employer{l}\location{r}\\
\body\\
\end{format}

%\title{Graduate Student}
%\employer{Institute of Computer Science}
%\location{University of the Philippines Los Ba\~{n}os}
%\dates{2002--Present}
%\begin{position}
Research interests include multiagent systems, distributed systems
, operating systems, artificial intelligence, data structures and algorithms
, data communications and networking, parallel and distributed computing, and computer security.
%\end{position}

%---------------------------------------------------------------------------

\section{\bf Teaching}

\begin{itemize}
%\item Mathematics Tutor, North Carolina State University, 2001
\item Assistant Professor, Computer Science Courses (Undergraduate),
Institute of Computer Science, University of the Philippines Los Ba\~{n}os, 2008 - present
\item Instructor, Computer Science Courses (Undergraduate),
Institute of Computer Science, University of the Philippines Los Ba\~{n}os, 2002 - 2008
\item Instructor, Linux Training (UPLB Employees), 
Institute of Computer Science, University of the Philippines Los Ba\~{n}os, 2005
\item Instructor, Open Source Software Development using LAMP (FNRI employees), 
Food and Nutrition Institute, Department of Science and Technology, 2006
\item Instructor, Linux Training (UPOU Employees), 
University of the Philippines Open University, 2004
\item Instructor, Linux Training (E-Konek employees), 
E-Konek, 2004
\item Tutor, Programming in C (UPOU students), 
University of the Philippines Open University, 2003
\end{itemize}

%------------------------------------------------------------------------------

%\section{\bf Employment}
%
% \begin{format}
% \employer{l}\location{r}\\
% \title{l}\dates{r}\\
% \body\\
% \end{format}
%
% \title{Computer Technician}
% \employer{Hurley Technology Resources}
% \location{Jefferson, NC}
% \dates{1999--2000}
% \begin{position}
% Windows and UNIX networking for businesses and internet service providers,
% in- and out-of-house PC repair, sales, technical support, and database driven
% web site design. Supervisor: Cyrus Hurley.
% \end{position}

%------------------------------------------------------------------------------
\section{\bf Courses Taught}
Introduction to Personal Computing, Object-oriented Programming, Web Programming
, Data Structures, Programming Languages, Operating Systems, File Processing and Database Systems
, Introduction to Software Engineering, Machine Organization and Assembly Language Programming
, Data Communications and Networking, Computer Architecture, Introduction to the Internet
, Numerical and Symbolic Computation, Undergraduate Seminar on Systems

\section{\bf Publications}
Hermocilla, J.A.C. 2009. ICS-OS: A Kernel Programming Approach to Teaching Operating System Concepts. Philippine Information Technology Journal 2(2):25-30.

Pabico, J.P., Hermocilla, J.A.C., Galang, J.P.C., and De Sagun, C.D. 2008. Perceived Social Loafing in Undergraduate Software Engineering Teams. Philippine Information Technology Journal 1(2):22-28

Hermocilla, J.A.C. and Pabico, J.P. 2010. Identifying Catchment Areas near Selected Mountains in the Philippines using FlowViz, In Proceedings (CDROM) of the 10th Philippine Computing Science Congress (PCSC 2010), Ateneo De Davao University, Davao City, 5-6 March 2010 (Submitted 24 January 2010; Accepted 22 February 2010; Presented 06 March 2010).

Hermocilla, J.A.C and Albacea, E.A. 2009. A Multiagent System Framework for Solving the Student Sectioning Problem (Extended Abstract), In Proceedings of 2009 Europe-Philippines International Workshop on Modeling, Simulation, and Grid Computing (MODEL 2009), Holy Name University, Tagbilaran City, Bohol, 04-06 May 2009 (Submitted 5 April 2009; Accepted 13 April 2009; Presented 6 May 2009)


\section{\bf Invited Talks}
\begin{itemize}
\item Open Source Revolution, Laguna State Polytechnic Colleges, Los Ba\~{n}os and
San Pablo Campus
\end{itemize}

\section{\bf Software Development}
\begin{itemize}
\item {\texttt{University Health Service Information System}}(2007-2009), A web application for automating the processes at the University Health Service of UPLB
\item {\texttt{Business Permit Acquisition System}}(2003-2005), A web application for processing business permits in the
Municipality of Los Ba\~{n}os, sponsored by the local government
\item {\texttt{GEMS-GCP Wrapper}}(2007), A Java application which is part of the Generation Challenge Program,
sponsored by IRRI
\end{itemize}

%\section{\bf Working Papers}

%``The Effects of Ties on Convergence in K-Modes Variants for Clustering
%Categorical Data,''
%with N.\ Orlowski, D.\ Schlorff, D.\ Ca\~{n}as, M.\ T.\ Chu,
%and R.\ E.\ Funderlic (2004).

%``Updating the Centroid Decomposition with Applications in LSI,''
%with M.\ T.\ Chu, D.\ Ca\~{n}as, R.\ E.\ Funderlic, N.\ Orlowski,
%and D.\ Schlorff (2004).

%``Structural Estimation of Sequential Games of Complete Information'' (2006).

%------------------------------------------------------------------------------

%%===========================================================================%%
%\newpage
%\opening

%\section{\bf Scientific Software}

%Blevins, J., J. Heath, and W. Hereman (2002):
%\href{http://www.mines.edu/fs_home/whereman/software/PDESolutionTester/}
%{\texttt{PDESolutionTester}},
%A Mathematica program for the symbolic verification of exact solutions of
%nonlinear partial differential equations.

%------------------------------------------------------------------------------

%\section{\bf Presentations}
%
%``Solving Nonlinear Wave Equations with Mathematica,''
%North Carolina State University, April 23, 2003.

%---------------------------------------------------------------------------

\section{\bf Technical Skills}

\begin{itemize}
\item Operating Systems: Linux, Windows 3.1/95/98/ME/XP, Solaris, DOS
\item Languages: C/C++, Java, PHP, HTML, XML, x86 Assembly Language, Javascript, PERL, Python,
UML, Visual Basic, SQL, LISP, COBOL
\item IDE: Microsoft Visual Studio, Eclipse, NetBeans
\item Database: MySQL, PostgresSQL, Oracle
\item Productivity Tools: OpenOffice, Microsoft Office
\item System/Network Administration: TCP/IP, DNS, FTP, Web Server, LDAP, Mail Server
\item Others: JSP, CVS, SVN
\end{itemize}

%%---------------------------------------------------------------------------%%

%\section{\bf Honors and Awards}
%\begin{itemize}
%\item \href{http://www.pbk.org/}{Phi Beta Kappa}, 2003
%\item Undergraduate Research Award, North Carolina State University, 2004
%\item Outstanding Teaching Assistant Award, Duke University, 2006
%\end{itemize}

%%---------------------------------------------------------------------------%%

%\begin{center}
%\vspace{\fill}\ \newline
%{\tiny \rm $ $RCSfile: cv-us.tex,v $ $ }
%{\tiny \rm $ $Date: 2007-10-28 16:25:37 +0800 (Sun, 28 Oct 2007) $ $ }
%{\tiny \rm $ $Revision: 1.28 $ $ }
%\end{center}

\end{resume}

\end{document}

%%===========================================================================%%
